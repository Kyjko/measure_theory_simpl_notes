\documentclass[xcolor=dvipsnames]{beamer}
\usepackage{amsfonts}
\usepackage{amsmath}
\usepackage{yfonts}
\usepackage{mathrsfs}
\usepackage{amssymb}
\usepackage[T1]{fontenc}
\usepackage{pgfplots}
\usepackage{pgfplotstable}
\usepackage{lmodern}
\usepackage{tikz}

\usetheme{CambridgeUS}

\author{BOGNÁR Miklós}
\title{Measure Theoretic View of Random Variables}
\begin{document}
\maketitle
\section{$\sigma$-algebras, measures and measurable functions}
\begin{frame}
\frametitle{$\sigma$-algebras, measures and measurable functions}
Let $T$ be a set. We will refer to it as the \emph{basis set}, on which we can construct our $\sigma$-algebra. $\sigma$ here refers to \emph{countable}. Let $\mathcal{S} \subseteq \mathcal{P}(T)$ the collection of subsets from $T$ with the property that it is closed under the $\sigma$-union, $\sigma$-intersection and complement operations, so for any $A_1, A_2, \dots \in \mathcal{S}$ we have
\[
	\bigcup_i^{\infty}{A_i} \in \mathcal{S}
\]
\[
	\bigcap_i^{\infty}{A_i} \in \mathcal{S}
\]
\[
	A_i^C \in \mathcal{S}
\]
We call $\mathcal{S}$ the \emph{$\sigma$-algebra over $T$}.
\end{frame}
\begin{frame}
We write the structure $(T, \mathcal{S})$ as a tuple, and we call it a \emph{measurable space}. A \emph{measure} is a function that assigns \emph{non-negative real numbers} to subsets of $T$:
\[
	\mu: \mathcal{P}(T) \mapsto \mathbb{R}^{+} \cup \{0\} 
\]
Note that \emph{not every possible subset of $T$ is measurable}, in fact, the ones that are measurable we call \emph{measurable sets}. A measure has to satisfy some additional properties, namely:
\[
	\mu(\emptyset) = 0
\]
\[
	\mu(\bigcup_{i=1}^{\infty}{E_i}) = \sum_{i=1}^{\infty}{\mu(E_i)}
\]
for disjoint sets $E_i \in \mathcal{S}$. We call the latter property $\sigma$-additivity.
\end{frame}
\begin{frame}
We call the structure $(T, \mathcal{S}, \mu)$ a \emph{measure space}. Note that in the case of non-disjoint sets, the $\sigma$-additivity becomes $\sigma$-subadditivity. If there is a set $F \in \mathcal{P}(T)$, then the $\sigma$-algebra \emph{generated by $F$} is the smallest $\sigma$-algebra $\sigma(F)$, such that it contains every set that is in $F$.
\end{frame}
\begin{frame}
Let $(X, \mathcal{S})$ and $(Y, \mathcal{T})$ be both measurable spaces. A function $f : X \mapsto Y$ is said to be \emph{measurable} if for any $E \in \mathcal{T}$, the pre-image of $E$ under $f$ is contained in $\mathcal{S}$, that is:
\[
	f^{-1}(E) \circeq \{x \in X : f(x) \in E\} \in \mathcal{S}
\]
We can write the function mapping in a way that emphasizes the $\sigma$-algebras:
\[
	f : (X, \mathcal{S}) \mapsto (Y, \mathcal{T})
\]
The $\sigma$-algebra \emph{generated by $f$}, denoted by $\sigma(f)$, is the set of pre-images
\[
	\sigma(f) \circeq \{f^{-1}(D) : D \in \mathcal{T}\}
\]
\end{frame}
\section{Probability space and measure}
\begin{frame}
\frametitle{Probability space and measure}
Let $\Omega$ be the set of all possible outcomes of a given experiment. We call $\Omega$ the \emph{sample space} or \emph{state space} of the experiment. Let $\mathcal{F} \subseteq \Omega$ be a $\sigma$-algebra over $\Omega$ such that every \emph{measurable} event is contained inside $\mathcal{F}$. Of course $\emptyset \in \mathcal{F}$, $\Omega \in \mathcal{F}$. Let us introduce the \emph{probability measure} $\mathbb{P}$ as the third and last member of the \emph{probability space} $(\Omega, \mathcal{F}, \mathbb{P})$. In order for $\mathbb{P}$ to be a valid probability measure, it has to fulfill the following requirements:
\[
	\mathbb{P}(\emptyset) = 0
\] 
\[
	\mathbb{P}(\Omega) = 1
\]
\[
	\forall A \in \mathcal{F}: \quad \mathbb{P}(A) \in [0, 1]
\]
We can therefore state that the measure $\mathbb{P}$ is a \emph{special type of measure} specifically used in aid of probability theory.
\end{frame}
\section{Random variables}
\begin{frame}
\frametitle{Random variables}
A \emph{measureable function} from the measurable space $(\Omega, \mathcal{F})$ to the real numbers equipped with the Borel $\sigma$-algebra $(\mathbb{R}, \mathcal{B}(\mathbb{R}))$ is called a \emph{Random variable}, often denoted by the capital $X$ or $Y$. The $\sigma$-algebra $\mathcal{F}$ is the algebra containing \emph{all the events that can have probabilities assigned to them}.
\[
	X : (\Omega, \mathcal{F}) \mapsto (\mathbb{R}, \mathcal{B}(\mathbb{R}))
\]
To make the notation easier, sometimes we write
\[
	X : \Omega \mapsto \mathbb{R}
\]
The $\sigma$-algebra \emph{generated by a random variable} $X$ is the set
\[
	\sigma(X) \circeq \{\omega \in \Omega : X(\omega) \in C, \quad \forall C \in \mathcal{B}(\mathbb{R})\}
\]
\end{frame}
\section{The distribution measure and function}
\begin{frame}
\frametitle{The distribution measure and function}
A random variable induces a probability measure on $(\mathbb{R}, \mathcal{B}(\mathbb{R}))$, often denoted by $\mu_X$, namely:
\[
	\mu_X(A) \circeq \mathbb{P}(X \in A), \quad \forall A \in \mathcal{B}(\mathbb{R})
\]
For a given random variable $X$, the \emph{distribution function} is the monotonically increasing function in the form
\[
	F_X(x) \circeq \mu_X((-\infty, x])
\]
essentially a shorthand notation when working with distribution measures, as it is almost always easier to consider the range of the random variable instead of its domain.
\end{frame}
\section{The density function}
\begin{frame}
\frametitle{The density function}
Let us now consider the distribution measure in the form of the \emph{integral measure}
\[
	\mu_X(E) = \int_{E}{g_Xd\lambda}, \quad \forall E \in \mathcal{F}
\]
where $g : \mathbb{R} \mapsto \mathbb{R}$ is a measurable function and $\lambda$ is the Lebesgue-measure on $\mathbb{R}$. We call $g_X$ the \emph{Radon-Nikodym derivative} of $\mu_X$ with respect to $\lambda$:
\[
	\frac{d\mu_X}{d\lambda} = g_X
\]
The notation here is symbolic, we can think of it as it we \emph{differentiated both sides} of the integral measure formula with respect to the measure $\lambda$.
In fact, $g$ is exactly the \emph{probability density function} of the random variable $X$. In the continuous real case, we can think of $g_X$ as being the derivative of the distribution function $F_X$ (if it exists):
\[
	g_X(x) = \frac{dF_X(x)}{dx}
\]
\end{frame}
\begin{frame}
although the measure-theoretical formalization is a lot less restrictive. Note that here the differentiation with respect to x is analogous with the differentiation with respect to $\lambda$, as the Lebesgue-measure can measure individual points. From the points mentioned above, it is also clear that $\forall x \in \mathbb{R}$:
\[
	F_X(x) = \int_{-\infty}^{x}{g_X(y)dy}
\]
\end{frame}

\section{Independence of random variables}
\begin{frame}
\frametitle{Independence of random variables}
Let us conside the probability space $(\Omega, \mathcal{F}, \mathbb{P})$, and let $\mathcal{A}$ and $\mathcal{B}$ be two \emph{sub-$\sigma$-algebras} of $\mathcal{F}$ ($\mathcal{A}, \mathcal{B} \subseteq \mathcal{F}$). We say that $\mathcal{A}$ and $\mathcal{B}$ are \emph{independent} if
\[
	\mathbb{P}(A \cap B) = \mathbb{P}(A)\mathbb{P}(B)
\]
for  $A \in \mathcal{A}, B \in \mathcal{B}$.
We say that two random variables are independent, if \emph{The $\sigma$-algebras generated by them are independent}.
\end{frame}

\section{Expectation and moments of random variables}
\begin{frame}
\frametitle{Expectation and moments of random variables}
The expectation (expected value) of a random variable $X$ is the linear operator in the form of an \emph{abstract integral} (\emph{Lebesgue-integral})
\[
	\mathbb{E}[X] \circeq \int_{\Omega}{Xd\mathbb{P}}
\]
where we integrate with respect to the probability measure $\mathbb{P}$.
If the random variable is in $L^2(\Omega, \mathcal{F}, \mathbb{P})$, meaning
\[
	\int_{\Omega}{X^2d\mathbb{P}} < \infty
\]
then the second moment of the random variable is
\[
	\mathbb{E}[X^2] \circeq \int_{\Omega}{X^2d\mathbb{P}}
\]
In general, the \emph{p-th moment} of $X$ is $\int_{\Omega}{X^pd\mathbb{P}}$ ($p>0$).
\end{frame}
\section{Variance of random variables}
\begin{frame}
\frametitle{Variance of random variables}
The \emph{variance} of a random variable is the expression
\[
	Var[X] \circeq \mathbb{E}[X^2] - \mathbb{E}^2[X]
\]
Note that in order for the variance to be finite, we need both the first and second moments to be finite as well.
\end{frame}

\end{document}